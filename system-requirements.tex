\chapter{System requirements}

\section{Using Overleaf}

Overleaf provides a rich-text editor so you don't need to know any code to get started—you can just edit the text, add images, and see the typeset document automatically created for you as you type. Our tutorial provides a quick three-step introduction to the main features. \\

Overleaf is an online collaborative writing and publishing tool that makes the whole process of writing, editing and publishing scientific documents much quicker and easier. Overleaf provides the convenience of an easy-to-use LaTeX editor with real-time collaboration and the fully compiled output produced automatically in the background as you type. \\

If you're familiar with LaTeX, using Overleaf couldn't be simpler as we provide full support for direct LaTeX editing, and automatically compile your document for you on our servers (so there's nothing to install). All you need to do is create a document and choose source mode in the editor to edit the LaTeX code for your paper. \\

\noindent The key features if Overleaf are :

\begin{itemize}
    \item All you need is a web browser
    \item Always have the latest version
    \item Effortless sharing
    \item Automatic real-time preview
    \item Real-time track changes and commenting
    \item Complementary Rich Text and LaTeX modes
    \item Quickly find LaTeX errors
\end{itemize}

If you're new to LaTeX and would like to learn more about it, we recommend completing our online introduction to LATEX course, prepared by Dr John Lees-Miller and originally presented at the University of Bristol. 

\section{Using MiKTEX and Texmaker}

\noindent The recommended TEX distributions are:
\begin{itemize}
    \item TEXLive is a major TEXdistribution for Unix/Linux, Mac OS and Windows.
    \item MiKTeX is a Windows-specific distribution.
    \item MacTeX is a Mac OS-specific distribution based on TEXLive.
\end{itemize}

The Texmaker is an editor with the text window, structure window, toolbars, functions, and status bar. The white portion shows the text or writing window, and the black part shows the structure window. It contains a link to chapters, sections, tables, equations, etc. The features of Texmaker are given below:

\begin{itemize}
    \item It includes spell checking while typing.
    \item It supports a variety of encodings.
    \item It contains a 'structure view,' which gets automatically updated while typing.
    \item In Texmaker, with the use of keyboard triggers, you can define an unlimited number of snippets.
    \item It gives you the full asymptotic support.
    \item It includes the built-in PDF viewer and the 'Quick Build' command.
    \item The Texmaker includes 37O mathematical symbols, which can be inserted in just one click.
    \item The extensive Latex document is furnished with the Texmaker.
    \item It automatically detects the warnings and errors with the corresponding line number after the compilation. It also contains the detail of each error.
    \item It also allows us to work efficiently onto the documents separated in several files with the "master mode."
\end{itemize}

